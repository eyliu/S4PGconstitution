\documentclass[12pt,letterpaper]{constitution}
\usepackage{mathpazo}
\usepackage{todonotes}
\usepackage{lineno}
\usepackage{xspace}
\usepackage{soul}
\usepackage{fancyhdr}
\usepackage{color}
\usepackage{graphicx}

%\newcommand{\temp}[1]{{\color{red} #1}}
\newcommand{\temp}[1]{\ul{\bf #1}\todo{TBD}\xspace}

\addtolength{\parskip}{\baselineskip}

\rhead{\thepage}
\chead{Article \Roman{article}}
\lhead{S4PG Constitution}
\rfoot{}
\cfoot{}
\lfoot{}

\begin{document}
\title{The Constitution of Students for Progressive Governance}
\author{\includegraphics[width=5in]{S4PGlogo}}
\date{Adopted: 1 Jan 2010}
\maketitle
\tableofcontents
\newpage

\pagestyle{fancy}
\headheight 35pt
\thispagestyle{empty}
\linenumbers

\begin{center}
	\bfseries Preamble
\end{center}

The student members of Students for Progressive Governance aim to solicit diverse opinions on the current structure of Student Government and organization at the University of Michigan -- Ann Arbor and to recommend changes to the current structure.


\article{Organization Name}
The organization will be known as Students for Progressive Governance or S4PG.


\article{Organization Membership}
\section{Eligibility.}
All students at the University of Michigan -- Ann Arbor campus are eligible to join Students for Progressive Governance.

\section{Membership Selection.}
Any current member can nominate new members during a general body meeting.  Upon a nomination and a second, the prospective member shall have a opportunity to address the General Body for a time determined by the Chair.  Following the prospective members address, the Chair shall facilitate discussion. Once the question is called, the nominated student shall become a member upon a simple majority vote.

\section{Voluntary Withdrawal.}
Any member can voluntarily withdraw at any point by informing the Chair, Vice Chair or Secretary.  

\section{Removal.}
A motion to removal is in order at any general body meeting with quorum. Upon a motion and second, the member in question will have the opportunity to speak for an amount of time determined by the chair which must be greater than three (3) minutes.  Following the defense, debate may take place.  Once the question is called, the member will be removed by a simple majority vote of the present and voting general body.

\subsection{Circumstances of Removal.}
A member may be removed for failure to comply with this constitution, dereliction of duty, criminal offense, verbal abuse or attempt to deliberately impede the work of S4PG.

\subsection{Recall of Officers.}
An officer may be recalled for any of the reasons outlined in section III.4.a. A motion to remove is in order at any general body meeting with quorum. Upon a motion and second, the member in question will have the opportunity to speak for an amount of time determined by the chair which must be greater than three (3) minutes.  Following the defense, debate may take place.  Once the question is called, the member will be removed by a simple majority  vote of the present and voting general body.


\article{Organizational Structure}
Students for Progressive Governance shall consist of a General Body and its sub committees and an Executive Council.

\section{The General Body.}
The General Body of Students for Progressive Governance consists of all current members.  The General Body may by a simple majority vote create or dispand internal sub- committees.  The General Body may veto any action of the Executive council by a simple majority vote. The General Body shall be following Robert’s Rules of Order (in its most current edition).  

\subsection{Internal Sub-Committees.}
The General Body may by a simple majority vote create or disband internal sub committees to aid with the internal S4PG initiatives.  Upon the creation of an internal committee there shall be an election within one (1) week for a chair of that committee.

\subsection{External Sub-Committees.}
The General Body may by a simple majority vote create or disband external committees to focus on educating the student body about the internal S4PG initiatives. Upon the creation of an external committee there shall be an election within one (1) week for a chair of that committee.

\section{The Executive Council.}
The Executive Council shall consist of the three executive officers and all Committee Chairs.  The executive council shall act have the ability to act on behalf of the General Body in emergencies until a General Body meeting may be called.   The Executive Council shall set the agenda for General Body meetings.  


\article{Organization Operations}
\section{Officer Elections.}
All elections shall occur at a general body meeting with quorum.  All elections for the Chair, Vice Chair and Secretary shall occur at the beginning of each academic year.  Any current member of Students for Progressive Governance is eligible to run for officer positions.  If the Chair is not available to preside over the meeting, the General Body shall appoint an Interim Chair until a Chair is elected.  All elections shall follow Robert’s Rules of Oder (in its most current edition).

\section{General Body Meetings.}
The Chair shall preside over all General Body meetings and shall uses Robert’s Rules of Order (in its most current edition) when during contested decision making.  General body meetings must be announced one (1) week prior.  Quorum for General Body meetings is 2/3 of organizational membership.

\section{Committee Meetings.}
The Committee Chair shall preside over their respective committee meeting in any way the Committee Chair deems fit.


\article{Officer Responsibilities}
\section{The Chair.}
The Chair is the chief executive officer of S4PG.  The Chair’s responsible for setting and holding meetings.  The Chair shall preside over all general body meetings with one (1) vote.  The Chair shall also have the tiebreaking vote on the Executive Council.  The Chair and shall oversee and coordinate outreach efforts of the organization. The Chair shall coordinate all press releases and media communication.

\section{The Vice Chair.}
The Vice Chair is a member of S4PG with one (1) vote in the General Body and Executive Council. The Vice Chair is responsible for overseeing the internal operations of Students 4 Progressive Governance, including but not limited to facilitating cross committee collaboration.  In the absence of the Chair, the Vice Chair shall assume all of the Chairs duties and responsibilities.

\section{The Secretary.}
The Secretary is a member of S4PG with one (1) vote in the General Body and Executive Council.  The Secretary shall keep and call roll at general body meetings, keep up to date minutes, insure efficient communication throughout the organization and shall insure that all internal documents are kept up to date.  Furthermore, the Secretary shall coordinate the drafting efforts of S4PG.

\section{The Committee Chair(s).}
The Committee Chairs must be members of S4PG.  They are responsible for presiding over their respective committee meetings and reporting back to the General Body. A Committee Chair shall have one (1) vote on the Executive Council. The Committee Chair shall choose a Vice Chair with the advice and consent of the General Body.

\subsection{The Committee Vice Chair(s).}
The Committee Vice Chairs are responsible for keeping records of the committee meetings and reporting back to the General Body.


\article{Organization Finances}
\section{Source of Funding.}
Students for Progressive Government shall seek funding grants from any University sources that the Executive Committee deems appropriate.  These sources include funding bodies from the various student governments on campus.

\section{Funding Practices.}
The Chair, Vice Chair and Secretary shall be the authorized signers on the SOAS account.  A disbursement may be made by any executive member pending the approval of the Executive Council and General Body.


\article{Amendments}
Amendments may be made to this student organization constitution by a two-thirds (2/3) vote of the General Body.  Motions to amend are in order whenever the General Body session is called to order.


\article{Registration Renewal}
S4PG shall be renewed once each year during the month of September and is the responsibility of the incoming Chair.
\end{document}
